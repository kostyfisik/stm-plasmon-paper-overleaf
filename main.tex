% ****** Start of file apssamp.tex ******
%
%   This file is part of the APS files in the REVTeX 4.1 distribution.
%   Version 4.1r of REVTeX, August 2010
%
%   Copyright (c) 2009, 2010 The American Physical Society.
%
%   See the REVTeX 4 README file for restrictions and more information.
%
% TeX'ing this file requires that you have AMS-LaTeX 2.0 installed
% as well as the rest of the prerequisites for REVTeX 4.1
%
% See the REVTeX 4 README file
% It also requires running BibTeX. The commands are as follows:
%
%  1)  latex apssamp.tex
%  2)  bibtex apssamp
%  3)  latex apssamp.tex
%  4)  latex apssamp.tex
%
\documentclass[%
 reprint,
%superscriptaddress,
%groupedaddress,
%unsortedaddress,
%runinaddress,
%frontmatterverbose, 
%preprint,
%showpacs,preprintnumbers,
%nofootinbib,
%nobibnotes,
%bibnotes,
 amsmath,amssymb,
%aps,
%pra,
prb,
%rmp,
%prstab,
%prstper,
%floatfix,
]{revtex4-1}

\usepackage{graphicx}% Include figure files
\usepackage{dcolumn}% Align table columns on decimal point
\usepackage{bm}% bold math
%\usepackage{hyperref}% add hypertext capabilities
%\usepackage[mathlines]{lineno}% Enable numbering of text and display math
%\linenumbers\relax % Commence numbering lines

%\usepackage[showframe,%Uncomment any one of the following lines to test 
%%scale=0.7, marginratio={1:1, 2:3}, ignoreall,% default settings
%%text={7in,10in},centering,
%%margin=1.5in,
%%total={6.5in,8.75in}, top=1.2in, left=0.9in, includefoot,
%%height=10in,a5paper,hmargin={3cm,0.8in},
%]{geometry}

\begin{document}

\preprint{APS/123-QED}

\title{Manuscript Title}% Force line breaks with \\
\thanks{A footnote to the article title}%

\author{Ann Author}
 \altaffiliation[Also at ]{Physics Department, XYZ University.}%Lines break automatically or can be forced with \\
\author{Second Author}%
 \email{Second.Author@institution.edu}
\affiliation{%
 Authors' institution and/or address\\
 This line break forced with \textbackslash\textbackslash
}%

\collaboration{MUSO Collaboration}%\noaffiliation

\author{Charlie Author}
 \homepage{http://www.Second.institution.edu/~Charlie.Author}
\affiliation{
 Second institution and/or address\\
 This line break forced% with \\
}%
\affiliation{
 Third institution, the second for Charlie Author
}%
\author{Delta Author}
\affiliation{%
 Authors' institution and/or address\\
 This line break forced with \textbackslash\textbackslash
}%

\collaboration{CLEO Collaboration}%\noaffiliation

\date{\today}% It is always \today, today,
             %  but any date may be explicitly specified

\begin{abstract}
An article usually includes an abstract, a concise summary of the work
covered at length in the main body of the article. 
\begin{description}
\item[Usage]
Secondary publications and information retrieval purposes.
\item[PACS numbers]
May be entered using the \verb+\pacs{#1}+ command.
\item[Structure]
You may use the \texttt{description} environment to structure your abstract;
use the optional argument of the \verb+\item+ command to give the category of each item. 
\end{description}
\end{abstract}

\pacs{Valid PACS appear here}% PACS, the Physics and Astronomy
                             % Classification Scheme.
%\keywords{Suggested keywords}%Use showkeys class option if keyword
                              %display desired
\maketitle

%\tableofcontents

\section{\label{sec:intro}Introduction}



Phenomenon of tunnel resonance and production of photons in the tunnel gap has been known for several decades [Lambec]. Research in this region can create prerequisites for realizing of operating circuits of electron-optical chips in which signals can be transmitted by photons, while the overall control of the circuit can be carried out by electrons. Such circuits provide a significant performance compared with traditional electronic circuits, in which all the processes of transmission and control are carried out by electrons.

For the implementation of such circuits, creation of planar sources of photons or plasmons controlled by electrons is required. One example of such sources are classical semiconductor lasers with Fabry-Perrot resonators or micro-ring resonators [Zhukov]. However, their dimensions cannot be less than micron units, which, from a practical point of view, substantially limits their use.

Submicron photon sources can be exemplified by the tunnel gap in STM [Elizabeth] or planar nanocontact [Bouhilier]. However, the efficiency of producing photons or plasmons is extremely low $10^{-6}-10^{-4}$.

To enhance this effect, optical nanoantennas [Krasnok] can be used. Usually, optical antennas operate in a light in/light out system. In case of locating optical nanoantenna in the tunnel junction region, it becomes possible to excite the antenna by tunnel electrons, which leads to an increase in the efficiency of photons production by several orders of magnitude [Hecht]. In this case, the source of visible or near-IR photons, controlled by electrical signals, can have subwave dimensions.

Efficiency of photons or plasmons production with the help of a nanocontact is a superposition of two phenomena: (1) the quantum process of inelastic electron tunnelling between the two sides of contact, to which the electric potential difference is applied, leading to the excitation of the optical mode in the contact region being a nanoantenna, and (2) energy relaxation of nanoantenna optical mode and emission of a photon or surface plasmon [Novotny].

  (1)
  
A convenient object for investigating a source of photons or plasmons controlled by electrons is the STM tip contact with the conductive substrate. When the STM tip approaches the metal substrate, due to the image potentials and redistribution of the electron density, a "reflection" of the tip is formed in the conductive substrate, which in turn can be considered as a dimer of two metallic nanoantennas, with localization of the electromagnetic field in the tunnel gap region. Thus, in the nanocontact region of the STM tip and the conductive substrate, the density of optical states (LDOS) for photons is increased by several orders in comparison with LDOS for a vacuum [Novotny]. This circumstance significantly influences , by increasing the quantum efficiency of the production of photons by electrons.

As shown in [Greffet], localization of the metallic nanoantenna in the region of the tunnel junction of the STM tip and the metal substrate dramatically affects the production efficiency of surface plasmons at the interface between the conductive substrate and air. On the one hand, the presence of a plasmon nanoantenna leads to an additional increase in LDOS, and as a result , in comparison with the case without a nanoantenna, and on the other hand, an excited plasmon nanoantenna significantly increases the energy conversion efficiency from the optical mode of the tunnel junction to the surface plasmon wave energy . It should be noted that due to ohmic losses, which are always inherent in metallic nanoantennas, some of the optical mode energy is expended on the Joule loss and heating of the nanocontact region.

In this paper, we consider the energy conversion efficiency of tunnel electrons of a nanocontact, into the tunnel gap of which a metallic plasmon nanoantenna or dielectric nanoantenna from a material with a high refraction index is placed. It is shown that both types of nanoantennas (plasmon and dielectric) substantially increase the production efficiency of surface plasmon waves at the conductive substrate-air interface. The dielectric nanoantenna has a greater efficiency than the metallic one, which makes this type of nanoantennas more promising when realistic electro-optic circuits are implemented.



\section{\label{sec:numerical}Numerical modelling}
Fig. 1 shows the general pattern of contact geometries in question: (1) metal tip – metal substrate, (2) metal tip – metal nanoantenna – metal substrate, (3) metal tip – dielectric nanoantenna – metal substrate. To simplify the problem under consideration, the STM tip was considered as a truncated cone, and the source of tunnel electrons was replaced by a point optical dipole. This approach is well proven in [Greffet].

It is clear that the shape of the STM tip should influence the nanocontact LDOS and the efficiency of the SPP production. The probe shape should not make any significant changes to the qualitative pattern of the processes; therefore, we have not studied the probe shape influence in this work.

\section{\label{sec:results}Results and discussions}

	As shown by the presented results of numerical simulation, the introduction of a metal or dielectric nanoantenna into the nanocontact region substantially increases the efficiency of energy conversion of inelastically tunnelling electrons into the energy of a surface plasmon wave. Moreover, although the total amount of SPP energy (measured in W) in the case of a metallic antenna is higher, the efficiency of the dielectric nanoantenna itself is higher than the metallic one. This circumstance is explained by the fact that the total amount of energy passing from the tunnel electrons to the SPP energy is a superposition of efficiency of conversion into the electro-optical mode  and the efficiency of the nanoantenna  itself (see Expression 1). The influence of both terms of expression (1) on the overall efficiency of SPP production by two types of nanoantennas is shown in Fig. Term for a metallic antenna exceeds that for a dielectric antenna, which determines the total energy of the optical mode excited by tunnel electrons, while the efficiency of the metallic antenna is lower due to ohmic losses.

Fig.  

Although in the case of metallic nanoantennas, the total energy passing to the SPP energy is larger, the use of dielectric nanoantennas may prove more promising in the creation of real electro-optical circuits. The presence of ohmic losses in metallic nanostructures leads to an additional heating of the region around them, and in the case of the geometry under consideration, to heating the nanocontact region. This increase in temperature leads to a change in the dimensions of nanocontact elements, and consequently to a change in the tunnel gap, instability of the nanocontact and tunnel current. As is well known, a change in the tunnel gap dimensions in the STM by only 1 A leads to a change in the tunnel current by 1 order. In the case of STM, these variations in the tunnel gap dimensions are processed by the microscope feedback loop: the probe either approaches or leaves the substrate, maintaining the tunnel current magnitude. To implement compact electro-optical circuits, the nanocontact must be planar; in this case the introduction of a feedback mechanism that maintains the magnitude of tunnelling current between the nanocontact sides is not very realistic. This circumstance imposes restrictions on the use of metallic nanoantennas. In dielectric nanoantennas, ohmic losses at Joule heating are absent, which makes the heat balance in the nanocontact region more stable and, therefore, the more stable tunnel gap and tunnel current.
\end{document}
%
% ****** End of file apssamp.tex ******